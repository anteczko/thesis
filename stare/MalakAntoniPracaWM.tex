\documentclass[12pt]{article}
\usepackage[a4paper]{geometry}
\usepackage{listings}
\usepackage{lingmacros}
\usepackage{graphicx}
\usepackage{tree-dvips}
\usepackage{polski}
\usepackage[utf8]{inputenc}
\usepackage[export]{adjustbox}% http://ctan.org/pkg/adjustbox
\let\oldincludegraphics\includegraphics
\renewcommand{\includegraphics}[2][]{%
  \oldincludegraphics[#1,max width=\linewidth]{#2}}
%\ProvidesPackage{utp-report}[UTP Report]
\RequirePackage{graphicx}
\RequirePackage{array}
\RequirePackage{sectsty}
\RequirePackage{xcolor}
\RequirePackage{listings}

\renewcommand\familydefault{\sfdefault}


\lstdefinestyle{custom}{
  belowcaptionskip=1\baselineskip,
  breaklines=true,
  showstringspaces=false,
  basicstyle=\footnotesize\ttfamily,
  keywordstyle=\bfseries\color{green!40!black},
  commentstyle=\itshape\color{purple!40!black},
  identifierstyle=\color{blue},
  stringstyle=\color{orange},
}
\lstset{escapechar=@,style=custom}

\definecolor{darkgray}{rgb}{0.3,0.3,0.3}
\subsubsectionfont{\color{darkgray}}

\newcommand{\screen}[1]{{\centering\includegraphics[width=10cm]{#1}}}

\newcommand\hr{\par\vspace{-.5\ht\strutbox}\noindent\hrulefill\par}
\newcommand{\B}[1]{\textbf{#1}}



\newcommand{\university}{\footnotesize Uniwersytet Technologiczno-Przyrodniczy}
\newcommand{\universityFill}{\footnotesize im. J.J. Śniadeckich w Bydgoszczy}
\newcommand{\department}{\footnotesize Wydział Telekomunikacji, Informatyki i Elektrotechniki}
\newcommand{\departmentFill}{\footnotesize Zakład Techniki Cyfrowej}

\newcommand{\universityFullName}{\def\arraystretch{1.4} \begin{tabular}{c} \university \\ \universityFill \\ \department \\ \departmentFill \end{tabular} \def\arraystretch{2}}

\newcommand\cimage[2][]{\raisebox{-0.4\height}{\includegraphics[#1]{#2}}}

\newcommand{\tabelka}{
\begin{table}[]
	\def\arraystretch{2}
	\begin{center}
		\begin{tabular}{|l|l|l|l|l|l|}
			\hline
			\multicolumn{1}{|c|}{\cimage[height=.9in]{utp}}             & \multicolumn{4}{c|}{\universityFullName}                                                                              & \multicolumn{1}{c|}{\cimage[height=.7in]{wtie}} \\ \hline
			\textbf{\footnotesize Przedmiot}                            & \multicolumn{3}{l|}{\footnotesize \project}                                    & \footnotesize \textbf{Kierunek/Tryb} & \footnotesize IS/NP                             \\ \hline
			\textbf{\footnotesize Numer lab.}                           & \footnotesize \title & \B{\footnotesize Data wykonania} & \footnotesize \date & \B{\footnotesize Data oddania}       & \footnotesize \dateEnd                         \\ \hline
			\textbf{\footnotesize Imię i nazwisko}                      & \multicolumn{3}{l|}{\footnotesize \author}                                     & \footnotesize \textbf{Grupa}         & \footnotesize \grupa                                 \\ \hline
		\end{tabular}
	\end{center}
\end{table}
}
%\endinput

\lstset{language=Java} 
\graphicspath{ {./scrots/} }

\renewcommand{\title}{01}
\newcommand{\project}{Wstępny szkic tematu pracy dyplomowej}
\renewcommand{\author}{Antoni Malak} 
\renewcommand{\date}{03.09.2021}
\newcommand{\dateEnd}{03.09.2021}
\newcommand{\grupa}{1}
\renewcommand{\thesection}{}
\renewcommand{\thesubsection}{\arabic{subsection}}
\renewcommand{\thesubsection}{}
\renewcommand{\thesubsubsection}{\arabic{subsubsection}}
\setcounter{section}{1}
\begin{document}
\tabelka


\section*{Zaprojektowanie i zaprojektowanie kafelkowego menadżera okien dla systemu Linux}

\subsubsection{Przegląd różnych typów menadżerów okien i ich funkcjonalności}
Celem projektu jest stworzenie nowego kafelkowego menadżera okien, który były w jakiś sposób unikalny, oraz posiadał funkcje, których nie ma w już istniejących programach tego typu. Dlatego ważne jest rozeznanie się w zakresie funkcji, które te programy oferują.
\\

\subsubsection{Określenie ogólnych funkcjonalności menadżera okien, oraz unikalnych cech do implementacji}
Menadżer okien określa zachowanie oraz pozycjonowanie okien programów. Może on również dodawać 'wirtualne pulpity/przestrzenie robocze' (eng. Workspaces), oraz definiować interakcję pomiędzy oknami programów.
\\
\subsubsection{Określenie użytych algorytmów, oraz ich przeznaczenia}
Menadżery przechowują okna w strukturze danych na różne sposoby, zazwyczaj jest to struktura drzewa, albo lista.
\\
\subsubsection{Struktura projektu oraz określenie jego części}
Wymagane w projekcie byłaby część, która zarządzałaby oknami programów, przechowywała o nich informacje, oraz część dzięki której użytkownik będzie mógł wchodzić w interakcję z programem/
\\
\subsubsection{Użyte biblioteki}
Menadżer okien jedynie zarządza oknami programów, ich rysowanie odbywa się dzięki zaimportowanym bilbiotekom graficznym.
\\
\subsubsection{Działanie oraz porówanie programu, do już istniejących, tego samego typu}
Końcowe porównanie progrmu wynikowego, do innych istniejących już rozwiązań, oraz pokazanie, że stworzenie jeszcze jednego nowego menadżera okien jednak miało sens.\
\\


\end{document}
