%\ProvidesPackage{utp-report}[UTP Report]
\RequirePackage{graphicx}
\RequirePackage{array}
\RequirePackage{sectsty}
\RequirePackage{xcolor}
\RequirePackage{listings}

\renewcommand\familydefault{\sfdefault}


\lstdefinestyle{custom}{
  belowcaptionskip=1\baselineskip,
  breaklines=true,
  showstringspaces=false,
  basicstyle=\footnotesize\ttfamily,
  keywordstyle=\bfseries\color{green!40!black},
  commentstyle=\itshape\color{purple!40!black},
  identifierstyle=\color{blue},
  stringstyle=\color{orange},
}
\lstset{escapechar=@,style=custom}

\definecolor{darkgray}{rgb}{0.3,0.3,0.3}
\subsubsectionfont{\color{darkgray}}

\newcommand{\screen}[1]{{\centering\includegraphics[width=10cm]{#1}}}

\newcommand\hr{\par\vspace{-.5\ht\strutbox}\noindent\hrulefill\par}
\newcommand{\B}[1]{\textbf{#1}}



\newcommand{\university}{\footnotesize Uniwersytet Technologiczno-Przyrodniczy}
\newcommand{\universityFill}{\footnotesize im. J.J. Śniadeckich w Bydgoszczy}
\newcommand{\department}{\footnotesize Wydział Telekomunikacji, Informatyki i Elektrotechniki}
\newcommand{\departmentFill}{\footnotesize Zakład Techniki Cyfrowej}

\newcommand{\universityFullName}{\def\arraystretch{1.4} \begin{tabular}{c} \university \\ \universityFill \\ \department \\ \departmentFill \end{tabular} \def\arraystretch{2}}

\newcommand\cimage[2][]{\raisebox{-0.4\height}{\includegraphics[#1]{#2}}}

\newcommand{\tabelka}{
\begin{table}[]
	\def\arraystretch{2}
	\begin{center}
		\begin{tabular}{|l|l|l|l|l|l|}
			\hline
			\multicolumn{1}{|c|}{\cimage[height=.9in]{utp}}             & \multicolumn{4}{c|}{\universityFullName}                                                                              & \multicolumn{1}{c|}{\cimage[height=.7in]{wtie}} \\ \hline
			\textbf{\footnotesize Przedmiot}                            & \multicolumn{3}{l|}{\footnotesize \project}                                    & \footnotesize \textbf{Kierunek/Tryb} & \footnotesize IS/NP                             \\ \hline
			\textbf{\footnotesize Numer lab.}                           & \footnotesize \title & \B{\footnotesize Data wykonania} & \footnotesize \date & \B{\footnotesize Data oddania}       & \footnotesize \dateEnd                         \\ \hline
			\textbf{\footnotesize Imię i nazwisko}                      & \multicolumn{3}{l|}{\footnotesize \author}                                     & \footnotesize \textbf{Grupa}         & \footnotesize \grupa                                 \\ \hline
		\end{tabular}
	\end{center}
\end{table}
}
%\endinput
